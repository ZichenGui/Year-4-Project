\documentclass[10pt]{article}
\setlength{\parindent}{0em}
\setlength{\parskip}{1em}

\usepackage[n, advantage, operators, sets, adversary, landau, probability, notions, logic, ff, mm, primitives, events, complexity, asymptotics, keys] {cryptocode}

\begin{document}
\section{Problem formulation}
Let $D$ be a two-dimensional table that supports the following operations:
\begin{itemize}
\item \textbf{Insert: } add a new row to the table.
\item \textbf{Delete: } remove a row from the table.
\item \textbf{Lookup: } find rows in the table that contains some keyword given as the input to the \textit{lookup} function.
\end{itemize}

Further, we assume that $D$ has $n$ columns, with $S_i$ the set possible attributes in the $i$-th column. We call $D$ a database.

Our goal is to construct a cryptographic database $D$ that is secure when out-sourced: no dishonest third-party server should be able to decrypt the database. We also want the database to be efficient on the operations above. In particular, we want \textit{lookup($\cdot$)} to be sub-linear time.


\section{Constructions}
Without loss of generality, we assume that $D$ has $n-1$ columns of actual entries. The $n$-th column is an auxiliary column that indicates if the corresponding row is genuine or fake.

The message to be encrypted is denoted as $m = (m_1, m_2,..., m_{n-1})$. So $m_i$ is the plaintext for the $i$-th column for this particular message. As a short hand, we write $\text{Enc}(m, \pk) = (\text{Enc}(m_1, \pk), \text{Enc}(m_2, \pk),...,\text{Enc}(m_{n-1}, \pk))$ to be the encryption scheme Enc applied to the message under public key \pk.

We write $(D, C)$ to mean insertion of $C$ (as rows) into the database $D$, and $C \concat x$ to mean concatenation of column $x$ to $C$.

For the constructions below, we encrypt the first $n-1$ columns deterministically. The auxiliary column is encrypted using a probabilistic encryption scheme.

Let DE = ($\text{Kg}_1, \text{Enc}_1, \text{Dec}_1$) be the deterministic encryption scheme and PKE = ($\text{Kg}_2, \text{Enc}_2, \text{Dec}_2$) be the probabilistic encryption scheme. Let $rand(C)$ be a function that shuffles rows of $C$. We define the following encryption schemes for databases.

\subsection{Exponential-space construction}

\begin{pcvstack}[center]
\begin{pchstack}
\procedure[linenumbering]{Key Generation$(1^n)$}{%
(\pk_1, \sk_1) \gets \text{Kg}_1(1^n) \\
(\pk_2, \sk_2) \gets \text{Kg}_2(1^n) \\
\pk \gets (\pk_1, \pk_2) \\
\sk \gets (\sk_1, \sk_2)
}

\pchspace
\procedure[linenumbering]{Insert$(m)$}{%
(\pk_1, \pk_2) \gets \pk \\
\pcfor x \in S_1 \times S_2 \times ... \times S_{n-1} \\
\t \pcif x = m \\
\t \t D \gets (D, (\text{Enc}_1(m, \pk_1) \concat \text{Enc}_2(True, \pk_2))) \\
\t \pcelse \\
\t \t D \gets (D, (\text{Enc}_1(x, \pk_1) \concat \text{Enc}_2(False, \pk_2)))
}
\end{pchstack}

\pcvspace
\begin{pchstack}
\procedure[linenumbering]{Decrypt$(D)$}{%
(\sk_1, \sk_2) \gets \sk \\
m \gets () \\
\pcfor c \text{ in } D \\
\t \pcparse c \text{ as } \bar{c} || x \\
\t \pcif \text{Dec}_2(x, \sk_2) = True \\
\t \t m \gets (m, \text{Dec}_1(c)) \\
\pcreturn m
}

\pchspace
\procedure[linenumbering]{lookup$(c, i)$}{%
r \gets () \\
\pcfor c \text{ in } D \\
\t \pcif c_i = c \\
\t \t r \gets (r,c) \\
\pcreturn r
}
\end{pchstack}
\end{pcvstack}




\section{Security notions}
\subsection{Indistinguishability of distributions}




\subsection{Indistinguishability of plaintext (1)}




\subsection{Indistinguishability of plaintext (2)}




\end{document}